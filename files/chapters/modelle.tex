\section{Modelle}
\begin{itemize}
\item Modelle sind Vereinfachte, bildliche oder mathematische Darstellungen von Strukturen, Funktionen und Verlaufsformen.
\item Modelle sind von den Menschen in den Grenzen ihrer Fortstellung entwickelt worden.
\item Sie sind der Wirklichkeit nur angenähert.
\item Modelle geben Teile der Wirklichkeit wieder -- aber nicht die \enquote{ganze Wirklichkeit}
\item Sie dienen dem Verständnis~\dots
\begin{itemize}
\item sie sind nicht wahr, oder falsch sondern sinnvoll
\end{itemize}
\end{itemize}
\subsection{Entwicklung des Atommodells}
\subsubsection{Leukipp und Demokrit: Atomistik}
500 v.\,Chr.: Natur besteht aus einer Vielzahl kleinster Teilchen, den Atomen
(vom griechischen atomos \enquote{unteilbar}).

\subsubsection{John Dalton}
Ende 18., Anfang 19. Jh.: Atome sind gleichmäßig von Stoff erfüllte Kugeln.

\subsubsection{Daraus resultierende Theorie: Stoßtheorie}
Atome befinden sich in regelloser Bewegung.\newline Sie stoßen zusammen:
\begin{itemize}
\item unwirksam\begin{itemize}
\item keine Reaktion (Teilchen bewegen sich nicht stark genug, oder fliegen wieder auseinander)
\end{itemize}
\item wirksam\begin{itemize}
\item Teilchen verbinden sich es kommt zur chemischen Reaktion
\end{itemize}\end{itemize}

\subsubsection{Joseph John Thomson}
19. Jh.: Atome, Kugeln bei denen Masse und positive Ladung (Protonen) über das gesamte Volumen verteilt sind, eingebetet
darin sind negative Teilchen (Elektronen)
\subsubsection{Ernest Rutherford: Streuversuch}
19. Jh.: bestrahlte Goldfolie mit Alpha-Teilchen (positiv geladene He-Ionen) von 100000 wurde nur eines abgelenkt.

\begin{picture}(30,28)
%\linethickness{1mm}
\put(40,8){\makebox(2,0){Hülle}}
\put(40,3){\makebox(2,0){$e^-$}}
\put(20,8){\makebox(1,0){Atomkern}}
\put(20,3){\makebox(1,0){$p^+$}}
\put(30,23){\makebox(0,0){Atom}}
\put(30,20){\line(1,-1){10}}%Senkrecht
\put(30,20){\line(-1,-1){10}}%Senkrecht
\put(100,15){\makebox(1,0){Anzahl $p^+$(Proton) \^= Anzahl $e^-$(Elektron)}}
\end{picture}

\subsubsection{Niels Bohr}
\begin{itemize}
\item will erklären, warum ein Atom nicht implodiert
\end{itemize}
\begin{picture}(30,28)
%\linethickness{1mm}
\put(40,8){\makebox(2,0){Hülle}}
\put(40,3){\makebox(2,0){$e^-$}}
\put(18,8){\makebox(1,0){Kern}}
\put(16,3){\makebox(1,0){$p^+$, Neutron}}
\put(30,23){\makebox(0,0){Atom}}
\put(30,20){\line(1,-1){10}}%Senkrecht
\put(30,20){\line(-1,-1){10}}%Senkrecht
\end{picture}
\begin{itemize}
\item $e^-$ bewegen sich auf kreis- oder ellipsenförmigen Bahnen \^= ihrer Energie um den Kern
\item diese Bahnen sind circa 200 mal größer als der Kern (dazwischen luftleerer Raum)
\item Elektronen der äußersten \enquote{Schale} haben die meiste Energie
%\section{Periodensystem der chemischen Elemente (PSE)}
%\begin{figure}[htbp]
%   \centering
%\includegraphics[scale=0.47]{periodensystem}
%   \caption{PSE}%
%\end{figure}
\end{itemize}
\subsection{\acf{PSE}}
Das \acf{PSE} entstand.

Die Namen der Elemente wurden anfänglich aus dem Griechischen abgeleitet, dann begann man die Elemente nach den Entdeckern des
entsprechenden Elements zu benennen, um diese zu ehren.

Im Allgemeinen aber stammen die Symbole der Elemente von deren lateinischen Namen ab.
